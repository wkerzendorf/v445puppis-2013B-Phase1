%%%%%%%%%%%%%%%%%%%%%%%%%%%%%%%%%%%%%%%%%%%%%%%%%%%%%%%%%%%%%%%%%%%%%%
%
%.IDENTIFICATION $Id: template.tex.src,v 1.41 2008/01/25 10:47:12 fsogni Exp $
%.LANGUAGE       TeX, LaTeX
%.ENVIRONMENT    ESOFORM
%.PURPOSE        Template application form for ESO Observing time.
%.AUTHOR         The Esoform Package is maintained by the Observing
%                Programmes Office (OPO) while the background software
%                is provided by the User Support System (USS) Department.
%
%-----------------------------------------------------------------------
%
%
%                   ESO LA SILLA PARANAL OBSERVATORY
%                   --------------------------------
%                   NORMAL PROGRAMME PHASE 1 TEMPLATE
%                   ---------------------------------
%
%
%
%          PLEASE CHECK THE ESOFORM USERS' MANUAL FOR DETAILED 
%              INFORMATION AND DESCRIPTIONS OF THE MACROS. 
%     (see the file usersmanual.tex provided in the ESOFORM package) 
%
%
%        ====>>>> TO BE SUBMITTED THROUGH WEB UPLOAD  <<<<====
%               (see the Call for Proposals for details)
%
%%%%%%%%%%%%%%%%%%%%%%%%%%%%%%%%%%%%%%%%%%%%%%%%%%%%%%%%%%%%%%%%%%%%%%

%%%%%%%%%%%%%%%%%%%%%%%%%%%%%%%%%%%%%%%%%%%%%%%%%%%%%%%%%%%%%%%%%%%%%%
%
%                      I M P O R T A N T    N O T E
%                      ----------------------------
%
% By submitting this proposal, the Principal Investigator takes full
% responsibility for the content of the proposal, in particular with
% regard to the names of CoI's and the agreement to act in accordance
% with the ESO policy and regulations, should observing time be
% granted.
%
%%%%%%%%%%%%%%%%%%%%%%%%%%%%%%%%%%%%%%%%%%%%%%%%%%%%%%%%%%%%%%%%%%%%%% 

%
%    - LaTeX *is* sensitive towards upper and lower case letters.
%    - Everything after a `%' character is taken as comments.
%    - DO NOT CHANGE ANY OF THE MACRO NAMES (words beginning with `\')
%    - DO NOT INSERT ANY TEXT OUTSIDE THE PROVIDED MACROS
%

%
%    - All parameters are checked at the verification or submission.
%    - Some parameters are also checked during the pdfLaTeX
%      compilation.  If this is not the case, this is indicated by the
%      phrase
%      "This parameter is NOT checked at the pdfLaTeX compilation."
%

\documentclass{esoform}

% The list of LaTeX definitions of commonly used astronomical symbols
% is already included in the style file common2e.sty (see Table 1 in
% the Users' Manual).  If you have your own macros or definitions,
% please insert them here, between the \documentclass{esoform}
% and the \begin{document} commands.
%
%     PLEASE USE NEITHER YOUR OWN MACROS NOR ANY TEX/LATEX MACROS  
%       IN THE \Title, \Abstract, \PI, \CoI, and \Target MACROS.
%
% WARNING: IT IS THE RESPONSIBILITY OF THE APPLICANTS TO STAY WITHIN THE
% CURRENT BOX LIMITS AND ELIMINATE POTENTIAL OVERFILL/OVERWRITE PROBLEMS 

\begin{document}

%%%%%%%%%%%%%%%%%%%%%%%%%%%%%%%%%%%%%%%%%%%%%%%%%%%%%%%%%%%%%%%%%%%%%%%%
%%%%% CONTENTS OF THE FIRST PAGE %%%%%%%%%%%%%%%%%%%%%%%%%%%%%%%%%%%%%%%
%%%%%%%%%%%%%%%%%%%%%%%%%%%%%%%%%%%%%%%%%%%%%%%%%%%%%%%%%%%%%%%%%%%%%%%%
%
%---- BOX 1 ------------------------------------------------------------
%
% You should use this template for period 92A applications ONLY.
%
% DO NOT EDIT THE MACRO BELOW. 

\Cycle{92A}

% Type below, within the curly braces {}, the title of your observing
% programme (up to two lines).
% This parameter is NOT checked at the pdfLaTeX compilation.
%
% DO NOT USE ANY TEX/LATEX MACROS IN THE TITLE

\Title{V445 Puppis - A unique opportunity to study helium novae}  

% Type below the numeric code corresponding to the subcategory of your
% programme.

\SubCategoryCode{D6}   

% Please specify the type of programme you are submitting. 
% Valid values: NORMAL, GTO, TOO, CALIBRATION, MONITORING
% If you specify TOO, you will also need to fill a ToO page below.
% If you specify CALIBRATION, then the SubCategory Code MUST be set to L0

% If your programme requires more than 100 hours the Large Programme
% template (templatelarge.tex) must be used.


\ProgrammeType{NORMAL}

% For GTO proposals only: uncomment the following and fill out the GTO
% programme code (as communicated to the respective GTO coordinator).

%\GTOcontract{INS-consortium}		

% For TOO proposals only: uncomment the following if you apply for
% Rapid Response Mode observations.
 
%\ObservationInRRM{}

% Uncomment the following macro if this proposal is applying for time
% under the VLT-XMM agreement (only available for odd periods).

%\ObservationWithXMM{}

%---- BOX 2 ------------------------------------------------------------
%
% Type below a concise abstract of your proposal (up to 9 lines).
% This parameter is NOT checked at the pdfLaTeX compilation.
%
% DO NOT USE ANY TEX/LATEX MACROS IN THE ABSTRACT

\Abstract{V445 Puppis was a unique nova.  Based on its slow lightcurve
 and spectra showing helium and carbon but not hydrogen, it was almost
 certainly the first known helium nova, i.e., a nova triggered by
 unstable helium burning.  Intriguingly, the inferred white-dwarf mass
 is close to the Chandrasekhar limit, making the system a plausible
 type Ia supernova progenitor.  Its nature, though, is unclear.  The
 simplest explanations involve some type of helium-star donor, but a
 white dwarf accreting and burning hydrogen could also build up a
 thick helium layer that ignites, and a veneer of hydrogen might
 escape notice.  In the first part of our project (in P90), we imaged
 the source using NACO and found that, as hoped, the dusty ejecta that
 had hidden the star for years had become sufficiently dilute to allow
 infrared studies of the system.  Here, we propose to do the second
 part, to try to determine the nature of the binary using NACO
 spectroscopy.}

%---- BOX 3 ------------------------------------------------------------
%
% Description of the observing run(s) necessary for the completion of
% your programme.  The macro takes ten parameters: run ID, period,
% instrument, time requested, month preference, moon requirement,
% seeing requirement, transparency requirement, observing mode and 
% run type.
%
% 1. RUN ID
% Valid values: A, B, ..., Z
% Please note that only one run per intrument is allowed for APEX
%
% 2. PERIOD
% Valid values: 92
% Exceptions:
% Monitoring Programmes: These programmes can span up to four periods.
%
% VLT-XMM proposals: These are only accepted in odd periods and are 
% also valid for the next period.
%
% This parameter is NOT checked at the pdfLaTeX compilation.
%
% 3. INSTRUMENT
% Valid values: AMBER CHAMPP CRIRES EFOSC2 FLAMES FLASH FORS2 HARPS HAWKI ISAAC KMOS LABOCA MIDI NACO OMEGACAM SABOCA SHFI SINFONI SOFI SOFOSC Special3.6 SpecialAPEX SpecialNTT SpecialUT2 SpecialVLTI UVES VIMOS VIRCAM XSHOOTER
% 
% Only Chilean and GTO Programmes are accepted on OMEGACAM.
% No normal programmes on OMEGACAM will be accepted.
% Please note that only a subset of these instruments will be accepted
% for Monitoring Programmes. Please see the Call for Proposals and the
% ESOFORM User Manual for more details.
%
% 4. TIME REQUESTED
% In hours for Service Mode, in nights for Visitor Mode.
% In either case the time can be rounded up to  1 decimal place. 
% This parameter is NOT checked at the pdfLaTeX compilation.
% 
% 5. MONTH PREFERENCE
% Valid values: oct, nov, dec, jan, feb, mar, any
%
% 6. MOON REQUIREMENT
% Valid values: d, g, n
%
% 7. SEEING REQUIREMENT
% Valid values: 0.4, 0.6, 0.8, 1.0, 1.2, 1.4, n
%
% 8. TRANSPARENCY REQUIREMENT
% Valid values: CLR, PHO, THN
%
% 9. OBSERVING MODE
% Valid values: v, s
%
% 10. RUN TYPE
% Valid values: TOO 
% For all Normal & Calibration Programmes this field should be blank.
% For TOO & GTO Programmes, users can specify TOO runs.
% If the field is left blank a default normal, non-TOO run is assumed.
% If a TOO run is specified please make sure that you fill in the TOO page.

\ObservingRun{A}{92}{NACO}{4h}{any}{n}{0.6}{CLR}{s}{}

% Proprietary time requested.
% Valid values: % 0, 1, 2, 6, 12

\ProprietaryTime{12}

%---- BOX 4 ------------------------------------------------------------
%
% Indicate below the telescope(s) and number of nights/hours already
% awarded to this programme, if any.
% This macro is optional and can be commented out.
% It is also NOT checked at the pdfLaTeX compilation.

\AwardedNights{VLT/NACO}{2h in 090.D-0112}

% Indicate below the telescope(s) and number of nights/hours still
% necessary, in the future, to complete this programme, if any.
% This macro is optional and can be commented out.
% It is also NOT checked at the pdfLaTeX compilation.

\FutureNights{}{}

%---- BOX 5 ------------------------------------------------------------
%
% Take advantage of this box to provide any special remark  (up to three
% lines). In case of coordinated observations with XMM, please specify
% both the ESO period and the preferred month for the XMM
% observations here.
% This macro is optional and can be commented out.
% It is also NOT checked at the pdfLaTeX compilation.

\SpecialRemarks{This is the follow-up promised in our P90 proposal for
  the case the binary had become detectable again.}
  
%---- BOX 6 ------------------------------------------------------------
% Please provide the ESO User Portal username for the Principal
% Investigator (PI) in the \PI field.
%
% For the Co-I's (CoI) please fill in the following details:
% First and middle initials, family name, the institute code
% corresponding to their affiliation. 
% The corresponding affiliation should be entered for EACH
% Co-I separately in order to ensure the correct details of 
% all Co-I's are stored in the ESO database.
% You can find all institute codes listed according to country
% on the following webpage:
% http://www.eso.org/sci/observing/phase1/countryselect.html
%
% For example, if the Co-I's full name is David Alan William Jones,
% his affiliation is the Observatoire de Paris, Site de Paris, 
% you should write:
% \CoI{D.A.W.}{Jones}{1588}
% Further examples are shown below.
% DO NOT USE ANY TEX/LATEX MACROS HERE
%

\PI{VANKERKWIJK}
\CoI{W.}{Kerzendorf}{1139}
%\CoI{K.}{Lepo}{2059}

% Please note: 
% Due to the way in which the proposal receiver system parses
% the CoI macro, the number of pairs of curly brackets '{}'
% in this macro MUST be strictly equal to 3, i.e., the
% number of parameters of the macro. Accordingly, curly
% brackets should not be used within the parameters (e.g.,
% to protect LaTeX signs).
%
% For instance:
% \CoI{L.}{Ma\c con}{1098}
% \CoI{R.}{Men\'endez}{1098}
%
% are valid, while
%
% \CoI{L.}{Ma{\c}con}{1098}
% \CoI{R.}{Men{\'}endez}{1098}
%
% are not. Unfortunately the receiver does not give an
% explicit error message when such invalid forms are
% used in the CoI macro, but the processing of the proposal
% keeps hanging indefinitely.


%%%%%%%%%%%%%%%%%%%%%%%%%%%%%%%%%%%%%%%%%%%%%%%%%%%%%%%%%%%%%%%%%%%%%%%%
%%%%% THE TWO PAGES OF THE SCIENTIFIC DESCRIPTION AND FIGURES %%%%%%%%%%
%%%%%%%%%%%%%%%%%&&&%%%%%%%%%%%%%%%%%%%%%%%%%%%%%%%%%%%%%%%%%%%%%%%%%%%%
%
%---- BOX 7 ------------------------------------------------------------
%
%               THIS DESCRIPTION IS RESTRICTED TO TWO PAGES 
%
%   THE RELATIVE LENGTHS OF EACH OF THE SECTIONS ARE VARIABLE,
%   BUT THEIR SUM (INCLUDING FIGURES & REFS.) IS RESTRICTED TO TWO PAGES
%
% All macros in this box are NOT checked at the pdfLaTeX compilation.

\ScientificRationale{\par
%
 \smallskip{\bf Bizarre lives of dead stars: white dwarfs in binaries.}
 While in isolation white dwarfs do little but cool steadily, in
 binaries they display an extremely varied phenomenology, ranging
 from very close direct-impact accretors with orbital periods as
 short as five minutes, to white dwarfs accreting from winds of giant
 companions in symbiotic systems with periods of years.  The final
 fate of white dwarfs in binaries is, theoretically at least,
 similarly varied: while a white dwarf may simply be left with a
 somewhat different mass, it can also undergo accretion-induced
 collapse to a neutron star, reincarnation as a helium giant star, or
 complete disintegration in a thermonuclear explosion known as a Type
 Ia supernova.

 Type Ia supernovae (SNe Ia) require the explosion of relatively
 massive white dwarfs.  Carbon fusion is usually thought to be
 ignited at very high density, in near pycno-nuclear conditions,
 reached when a white dwarf approaches the Chandrasekhar mass --
 either because of Roche-Lobe overflow from a non-degenerate
 companion (single-degenerate scenario), or a merger of two white
 dwarfs (double-degenerate scenario).

 Despite decades of study, however, the observational evidence remains
 confusing.  On the one hand, there is no sign of a non-degenerate
 companion in some supernovae (e.g., SN 2011fe, Nugent et al.\ 2011,
 Nature 480, 344; Bloom et al.\ 2012, ApJ 744, L17) and some supernova
 remnants (Schaefer \& Pagnotta, 2012, Nature, 481, 164; Kerzendorf et
 al.\ 2012, ApJ, Kerzendorf et al. 2013, submitted) as well as a lack
 of suitable single-degenerate progenitor systems from both
 observations (Di Stefano 2010, ApJ 712, 728; Gilfanov \& Bogd\'an
 2010, Nature 463, 924) and population synthesis (Ruiter et al.\ 2009,
 ApJ 699, 2026; Mennekens et al.\ 2010, A\&A 515, 89).  On the other
 hand, a fair fraction of all SN Ia show evidence for circumstellar
 medium that is most easily interpreted as winds from a companion or
 accretion disk (e.g., Patat et al. 2007, Science 317, 924; Sternberg
 et al.\ 2011, Science 333, 856).

 The above suggests that SNe Ia may be produced via multiple channels.
 E.g., in one less commonly considered scenario, which may contribute
 $\sim\!10$\% of the total (Ruiter et al., ibid), accretion is from a
 helium-rich donor in an AM CVn binary.  For all accretion scenarios,
 however, a general question is whether the accretion is efficient
 enough to produce massive white dwarfs (for a review, Hillebrandt
 and Niemayer 2000, ARA\&A 38, 191).

 \medskip{\bf Growth or erosion: the role of nova explosions.}
 A major uncertainty in the fate of slowly accreting white dwarfs, is
 whether they in fact gain mass.  As the accreted layer of hydrogen
 thickens, fusion eventually starts, leading to a nova, and
 concomittant mass loss.  Estimates of the ejecta masses as well as
 the clear presence of dredged up core material (overabundance of
 carbon and oxygen), suggest the white dwarf experiences, at best,
 slow growth (Townsley \& Bildsten 2004, ApJ 600, 390, and references
 therein).

 The above problem can be circumvented in two ways.  First, in sources
 that accrete relatively quickly, hydrogen is burnt steadily
 (observationally, these are the supersoft sources).  Alternatively,
 one can have a helium-rich donor (e.g., a helium white dwarf, as in
 the AM CVn systems mentioned above; a subdwarf; or a helium giant).
 For all above sources, however, there is another hurdle: as the layer
 of helium thickens, it in turn may ignite, leading to a {\em helium
   nova}.  The effect of these on growth has usually been ignored (but
 see Idan et al.\ 2012, J.~Ph.\ Conf.\ Ser.\ 337, 12051).  In part,
 this likely was because, until recently, no helium novae had been
 identified.

 \medskip{\bf The first helium nova: V445 Puppis.}
 The 2000 nova in Puppis was quickly found to be unusual, with
 spectra showing no sign of hydrogen but being unusually rich in
 carbon (Ashok \& Banerjee 2003, A\&A 409, 1007; Iijima \& Nakanishi
 2008, A\&A 482, 865), and a visual lightcurve dropping precipitously
 due to dust formation.  Modelling showed that it was consistent with
 a helium nova on a massive, likely $\ga\!1.35\,M_\odot$ white dwarf
 (Kato et al.\ 2008, ApJ 684, 1366), while NACO follow-up
 observations found it had led to a beautiful bipolar nebula with a
 thick equatorial waist (Woudt et al.\ 2009, ApJ 706, 738; also, ESO
 press release 0943).

 The pretty nebula also posed a problem, however: the equatorial waist
 obscured the binary (see figure).  As a result, the only clues to its
 nature were that pre-explosion V445 Pup had $V\simeq14.5$ and
 $V-K\simeq3.0$, and that it was variable at the 10\% level.  At the
 distance of $8.2\pm0.5\,$kpc inferred from the expansion of the
 nebula by Woudt et al., and correcting for interstellar reddening
 $E_{B-V}\simeq0.5$, the implied luminosity is $L\simeq2000\,L_\odot$
 and the intrinsic color $(V-K)_0\simeq1.6$ (or more luminous and
 bluer if there is substantial circumstellar reddening).

 The above luminosity is puzzlingly high for a cataclysmic variable.
 Could the companion be very luminous? (But it is too blue to be a
 symbiotic.) Or is the accretion disk very large?  Or might the system
 host a supersoft source?  Furthermore, does the absence of evidence
 for hydrogen reflect the absence of hydrogen, or an undetectable
 amount of hydrogen (as might happen in a supersoft source)?

 These questions have broader implications: if the accretion rate was
 such that the system had regular nova events, it implies that novae
 leave more mass than they remove, since otherwise no helium layer
 could build up.  If the system was a supersoft source instead, an
 estimate of the ejecta mass will provide a clue whether such sources
 grow in mass despite helium novae.  Furthermore, in both
 hydrogen-rich cases, V445 Pup would prove it is possible to hide
 hydrogen from sight in the explosion, removing one counterargument
 against those sources as SN Ia progenitors (that there is no
 evidence for hydrogen in SN Ia spectra).  If instead, the donor is a
 helium star or giant, the mass-loss and mass accretion estimates
 will allow a first empirical evaluation of the plausibility of such
 systems as SN Ia progenitors (and would suggest that regular novae
 do loose more mass than they accrete, since no earlier helium novae
 were observed).
%
}

\ImmediateObjective{%
%
 For our P90 proposal, we were granted 2 hours of NACO AO imaging to
 determine whether V445 Pup had become visible again, as we hoped
 might be the case given that the expansion of the nebula should have
 substantially decreased the dust column.  Our first observation,
 taken in last December, amply fulfilled our hopes (see figure): the
 source had re-appeared with $K\simeq16.5$.

 Having foreseen the possibility of being so lucky, we had requested
 the second one-hour observing block to be scheduled well after the
 first, so we might use it for spectroscopy instead of another image
 and thus get a first glimpse at the nature of the system.  While the
 resulting spectrum -- taken end of January -- is too noisy for
 classification (typical signal-to-noise of 10 per 1\,nm pixel), it
 suffices to show there are no strong spectral features.  The absence
 of strong emission features implies that the contribution from the
 nebula is minimal, and strongly suggests that the contribution from
 the accretion disk is small as well.  Hence, most likely the
 companion dominates and our ultimate goal of determining its nature
 is possible.

 Here, we proprose to obtain a better spectrum, suitable for
 classification.  Specifically, we aim for a final signal-to-noise
 ratio of 30, by integrating four times longer and using better
 conditions (and thus AO correction): 0.6 arcsec seeing (instead of
 0.8), and clear sky (instead of thin clouds - causing an absorption of 0.2 mag).  With this, we should
 be able to ascertain the nature of V445~Pup: Is the binary hydrogen
 or helium rich? How large and luminous is the companion? How large is
 the accretion disk? Did the system host a supersoft source?
%
}
%
%---- THE SECOND PAGE OF THE SCIENCE CASE CAN INCLUDE FIGURES ----------
%
% Up to ONE page of figures can be added to your proposal.  
% The text and figures of the scientific description must not
% exceed TWO pages in total. 
% If you use color figures, do make sure that they are still readable
% if printed in black and white. Figures must be in PDF or JPEG format.
% Each figure has a size limit of 1MB.
% MakePicture and MakeCaption are optional macros and can be commented out.

\MakePicture{v445pup_2007vs2012_cut.pdf}{angle=0,width=1.0\textwidth}
\MakeCaption{Fig.~1: NACO observations taken March 2007 by Woudt et
  al. (on the left) and our observations taken in December 2012 (on
  the right). The binary is now visible with $K\simeq16.5$.  (Note
  that we used very different stretch for the two figures, to show how
  much fainter the companion was in 2007.)}

%%%%%%%%%%%%%%%%%%%%%%%%%%%%%%%%%%%%%%%%%%%%%%%%%%%%%%%%%%%%%%%%%%%%%%
%%%%% THE PAGE OF TECHNICAL JUSTIFICATIONS %%%%%%%%%%%%%%%%%%%%%%%%%%%%%
%%%%%%%%%%%%%%%%%%%%%%%%%%%%%%%%%%%%%%%%%%%%%%%%%%%%%%%%%%%%%%%%%%%%%%%%
%
%---- BOX 8 ------------------------------------------------------------
%
% Provide below a careful justification of the requested lunar phase
% and of the requested number of nights or hours.  
% All macros in this box are NOT checked at the pdfLaTeX compilation.

\WhyLunarPhase{No constraints, since the observations are taken in the NIR.}  

\WhyNights{%
 To measure the Hydrogen/Helium emission lines as well as stellar parameters we need a spectrum with a S/N$\approx30$. 
 
 Our previous observations used the acquisition template `NACO\_img\_acq\_MoveToSlit'  with 00:17:41.00 and 4 coadds with 634.75~s each filling an OB with exactly one hour (to obtain an ABBA pattern). Which leaves  4 integration times of 634.75 s to fill an hour. This resulted in a S/N=10 and we have ascertained (using the ETC) that with better conditions (0.6\arcsec seeing and CLR conditions) we obtain a $\textrm{S/N}\approx15$. We can therefore reach our target S/N=30 with 16 NDITs of 634.75s, resulting in a total of 4 OBs with 1 hour each when taking overheads into account. 

}

\TelescopeJustification{V445 Puppis is currently enshrouded in dust,
and we expect extinctions $A_K \approx 4$ even during our observations.
Therefore, the NIR is the logical wavelength range.  We also need
high spatial resolution to separate the central object from the
relatively bright nova remnant.  NIR AO provides this.

NACO is also one of the only instruments that allows for a guide star that is separated by 23\arcsec. In particular,
for this reason, neither NICI on Gemini South nor Altair+NIRI on
Gemini North are suitable -- which is the reason for this submission
to ESO by proposers from Canada.  (The absence of nearby bright guide
stars is also one reason we did not propose for SINFONI, which might
have seemed a better instrument to try to tease out a spectrum from
the surrounding nebula.)

%
}

\ModeJustification{The requested observations do not need any special
calibration requirements and are short.  Service mode is therefore a
logical choice.}


% Please specify the type of calibrations needed.
% In case of special calibration the second parameter is used to enter 
% specific details.
% Valid values: standard, special
%\Calibrations{special}{Adopt a special calibration}
\Calibrations{standard}{}


%%%%%%%%%%%%%%%%%%%%%%%%%%%%%%%%%%%%%%%%%%%%%%%%%%%%%%%%%%%%%%%%%%%%%%%
%% PAGE OF BOXES 9-10  %%%%%%%%%%%%%%%%%%%%%%%%%%%%%%%%%%%%%%%%%%%%%%%%
%%%%%%%%%%%%%%%%%%%%%%%%%%%%%%%%%%%%%%%%%%%%%%%%%%%%%%%%%%%%%%%%%%%%%%%
%
%---- BOX 9 -- Use of ESO Facilities --------------------------------
%
% This macro is optional and can be commented out.
% It is also NOT checked at the pdfLaTeX compilation.
% LastObservationRemark: Report on the use of the ESO facilities during
%  the last 2 years (4 observing periods). Describe the status of the
%  data obtained and the scientific output generated.

\LastObservationRemark{
Van Kerkwijk, Kerzendorf and Lepo have obtained imaging and spectral observations for this object as described in our science justification.

Van Kerkwijk and Lepo obtained FLAMES data Fall
 2010 in the context of 386.D-0541(A), ``A hunt for supernova type Ia
 progenitors among optically faint, UV-bright stars in the Small
 Magellanic Cloud.'' The data are beautiful and have been fully
 reduced and analysed.  The manuscript is to be submitted shortly
 (after the current proposal deadlines have passed).}

%
%---- BOX 9a -- ESO Archive ------------------------------------------
%
% Are the data requested in this proposal in the ESO Archive
% (http://archive.eso.org)? If yes, explain the need for new data.
% This macro is NOT checked at the pdfLaTeX compilation.

\RequestedDataRemark{Data for V445 Puppis exists, but only now the central star is visible to take a spectrum.}

%
%---- BOX 9b -- ESO GTO/Public Survey Programme Duplications---------
%
% If any of the targets/regions in ongoing GTO Programmes and/or
% Public Surveys are being duplicated here, please explain why.
% This macro is optional and can be commented out.
% It is also NOT checked at the pdfLaTeX compilation.

\RequestedDuplicateRemark{The object is not being observed by surveys
or GTO programs.} 

%
%---- BOX 10 ------ Applicant(s) publications ---------------------
%
% Applicant's publications related to the subject of this proposal
% during the past two years.  Use the simplified abbreviations for
% references as in A&A.  Separate each reference with the following
% usual LaTex command: \smallskip\\
%   
%   Name1 A., Name2 B., 2001, ApJ, 518, 567: Title of article1
%   \smallskip\\
%   Name3 A., Name4 B., 2002, A\&A, 388, 17: Title of article2
%   \smallskip\\
%   Name5 A. et al., 2002, AJ, 118, 1567: Title of article3
%
% This macro is NOT checked at the pdfLaTeX compilation.

\Publications{
None of the applicants have previous experience with novae, but all
are interested in SN Ia progenitors (recent interest of Van Kerkwijk,
PhD thesis of Lepo):\medskip\\ 
Van Kerkwijk, M.H., Chang, P., Justham, S., 2010,
``Sub-Chandrasekhar White Dwarf Mergers as the Progenitors of Type
Ia Supernovae,'' ApJ 722, L157
\smallskip\\
Kerzendorf, W.E., Schmidt, B.P., Asplund, M., Nomoto, K.,
Podsiadlowski, Ph., Frebel, A., Fesen, R.A., Yong, D., 2009, ``Subaru
High-Resolution Spectroscopy of Star G in the Tycho Supernova
Remnant,'' ApJ 701, 1655.
\smallskip\\
Kerzendorf, W.E., Schmidt, B.P., Laird, J.B., Podsiadlowski, P.,
Bessell, M.S., 2012, ``Hunting for the progenitor of SN 1006: High
resolution spectroscopic search with the FLAMES instrument,'' ApJ,  759, 2012
\smallskip\\
Kerzendorf, Wolfgang E.; Yong, David; Schmidt, Brian P.; Simon, Joshua D.; Jeffery, C. Simon; Anderson, Jay; Podsiadlowski, Philipp; Gal-Yam, Avishay; Silverman, Jeffrey M.; Filippenko, Alexei V.; Nomoto, Ken'ichi; Murphy, Simon J.; Bessell, Michael S.; Venn, Kim A.; Foley, Ryan J.  ``A High-Resolution Spectroscopic Search for the Remaining Donor for Tycho's Supernova'', Apj, submitted


}


%%%%%%%%%%%%%%%%%%%%%%%%%%%%%%%%%%%%%%%%%%%%%%%%%%%%%%%%%%%%%%%%%%%%%%%%
%%%%% THE PAGE OF THE TARGET/FIELD LIST %%%%%%%%%%%%%%%%%%%%%%%%%%%%%%%%
%%%%%%%%%%%%%%%%%%%%%%%%%%%%%%%%%%%%%%%%%%%%%%%%%%%%%%%%%%%%%%%%%%%%%%%%
%
%---- BOX 11 -----------------------------------------------------------
%
% Complete list of targets/fields requested.  The macro takes nine
% parameters: run ID, target field/name, RA, Dec, time on target, magnitude, 
% diameter, additional information, reference star.
%
% 1. RUN ID
% Valid values: run IDs specified in BOX 3
%
% 2. TARGET FIELD/NAME
%
% 3. RA (J2000)
% Format: hh mm ss.f, or hh mm.f, or hh.f
% Use 00 00 00 for unknown coordinates
% This parameter is NOT checked at the pdfLaTeX compilation.
% 
% 4. Dec (J2000)
% Format: dd mm ss, or dd mm.f, or dd.f
% Use 00 00 00 for unknown coordinates
% This parameter is NOT checked at the pdfLaTeX compilation.
%
% 5. TIME ON TARGET
% Format: hours (overheads and calibration included)
% This parameter is NOT checked at the pdfLaTeX compilation.
%
% 6. MAGNITUDE
% This parameter is NOT checked at the pdfLaTeX compilation.
%
% 7. ANGULAR DIAMETER
% This parameter is NOT checked at the pdfLaTeX compilation.
%
% 8. ADDITIONAL INFORMATION
% Any relevant additional information may be inserted here.
% For APEX and CRIRES runs, the requested PWV upper limit MUST
% be specified for each target using this field.
% For APEX runs, the acceptable LST range MUST also be specified here.
% This parameter is NOT checked at the pdfLaTeX compilation.
%
% 9. REFERENCE STAR ID
% See Users' Manual.
% This parameter is NOT checked at the pdfLaTeX compilation.
%
% Long lists of targets will continue on the last page of the
% proposal.
%
%                       ** VERY IMPORTANT ** 
% The scheduling of your programme will take into account ALL targets
% given in this list. INCLUDE ONLY TARGETS REQUESTED FOR P92 !
% (except for VLT-XMM proposals)
%
% DO NOT USE ANY TEX/LATEX MACROS FOR THE TARGETS

\Target{A}{V445 Puppis}{07 37 56.90}{-25 56 58.9}{1.0}{16}{}{}{2MASS J07375835-2556462}


%                      ***************** 
%                      ** PWV limits **
% For CRIRES and all APEX instruments users must specify the PWV upper
% limits for each target. For example:
%\Target{}{Alpha Ori}{06 45 08.9}{-16 42 58}{1}{-1.4}{6 mas}{PWV=1.0mm, Sirius}{}
%\Target{}{HD 104237}{12 00 05.6}{-78 11 33}{1}{}{}{PWV<0.7mm;LST=9h00-15h00}{}
%
%                      *****************

% Use TargetNotes to include any comments that apply to several or all
% of your targets.
% This macro is NOT checked at the pdfLaTeX compilation.

\TargetNotes{NGS 2MASS J07375835-2556462 is GSC2.3 S3EQ038054 used
also by Woudt et al. (2009)}


%%%%%%%%%%%%%%%%%%%%%%%%%%%%%%%%%%%%%%%%%%%%%%%%%%%%%%%%%%%%%%%%%%%%%%%%
%%%%% TWO PAGES OF SCHEDULING REQUIREMENTS %%%%%%%%%%%%%%%%%%%%%%%%%%%%%
%%%%%%%%%%%%%%%%%%%%%%%%%%%%%%%%%%%%%%%%%%%%%%%%%%%%%%%%%%%%%%%%%%%%%%%%
%
%---- BOX 12 -----------------------------------------------------------
%

% Uncomment the following line if the proposal involves time-critical
% observations, or observations to be performed at specific time
% intervals. Please leave these brackets blank. Details of time
% constraints can be entered in Special Remarks and using the
% other flags in Box 13.
%
%
%\HasTimingConstraints{}

%
% The timing constraint macros listed below 
% are optional and can be commented out:
% \HasTimingConstraints, \RunSplitting, \Link and \TimeCritical
% They are also NOT checked at the pdfLaTeX compilation.


% 1. RUN SPLITTING, FOR A GIVEN ESO TELESCOPE (Visitor Mode only)
%
% 1st argument: run ID
% Valid values: run IDs specified in BOX 3
%
% 2nd argument: run splitting requested for sub-runs
% This parameter is NOT checked at the pdfLaTeX compilation.

%\RunSplitting{A}{1H,10w,1H}
%\RunSplitting{C}{2,10s,2,20w,2,15s,4H2}


% 2. LINK FOR COORDINATED OBSERVATIONS BETWEEN DIFFERENT RUNS.
%\Link{B}{after}{A}{10}
%\Link{C}{after}{B}{}
%\Link{E}{simultaneous}{F}{}

% 3. UNSUITABLE PERIOD(S) OF TIME
%
% 1st argument: run ID
% Valid values: run IDs specified in BOX 3
%
% 2nd argument: Chilean start date for the unsuitable time
% Format: dd-mmm-yyyy
% This parameter is NOT checked at the pdfLaTeX compilation.
%
% 3rd argument: Chilean end date for the unsuitable time
% Format: dd-mmm-yyyy
% This parameter is NOT checked at the pdfLaTeX compilation.

%\UnsuitableTimes{A}{15-jan-14}{18-jan-14}{Insert explanation of unsuitable time here.}
%\UnsuitableTimes{B}{15-jan-14}{18-jan-14}{Insert explanation of unsuitable time here.}
%\UnsuitableTimes{C}{20-jan-14}{23-jan-14}{Insert explanation of unsuitable time here.}


%
%---- BOX 12 contd.. -- Scheduling Requirements 
%

% SPECIFIC DATE(S) FOR TIME-CRITICAL OBSERVATIONS
% Please note: The dates must correspond to the LOCAL CHILEAN observing dates.
%
% The 2nd and 3rd parameters are NOT checked at the pdfLaTeX compilation.
% 1st argument: run ID
% Valid values: run IDs specified in BOX 3
%
% 2nd argument: Chilean start date for the critical period.
% Format: dd-mmmm-yyyy 
%
% 3rd argument: Chilean end date for the critical period.
% Format: dd-mmmm-yyyy

%\TimeCritical{A}{12-nov-13}{14-nov-13}{Insert reason for time-critical observations.}
%\TimeCritical{D}{1-nov-13}{2-nov-13}{Insert reason for time-critical observations.}
%\TimeCritical{D}{17-nov-13}{18-nov-13}{Insert reason for time-critical observations.}
%\TimeCritical{D}{23-nov-13}{24-nov-13}{Insert reason for time-critical observations.}



%%%%%%%%%%%%%%%%%%%%%%%%%%%%%%%%%%%%%%%%%%%%%%%%%%%%%%%%%%%%%%%%%%%%%%%%
%
%---- BOX 14 -----------------------------------------------------------
%
% INSTRUMENT CONFIGURATIONS:
%
% Uncomment only the lines related to instrument configuration(s)
% needed for the acquisition of your planned observations. 
%
% 1st argument: run ID
% Valid values: run IDs specified in BOX 3
%
% 2nd argument: instrument
% This parameter is NOT checked at the pdfLaTeX compilation.
%
% 3rd argument: mode
% This parameter is NOT checked at the pdfLaTeX compilation.
%
% 4th argument: additional information
% This parameter is NOT checked at the pdfLaTeX compilation.
%
% All parameters are mandatory and cannot be empty. Do NOT specify
% Instrument Configurations for alternative runs.

% Examples (to be commented or deleted)


%\INSconfig{A}{FORS2}{IMG}{ESO filters: provide list HERE}
%\INSconfig{B}{VIMOS}{IFU 0.33"/fibre}{LR-Blue}
%\INSconfig{C}{EFOSC2}{Imaging-filters}{EFOSC2 filters: provide list here}
%\INSconfig{D}{NACO}{IMG 54 mas/px VIS-WFS}{provide list of filters HERE}
%\INSconfig{E}{AMBER}{LR-HK}{2.2}
%\INSconfig{F}{MIDI}{PRISM}{HIGH-SENS}
%
% Real list of instrument configurations

%%%%%%%%%%%%%%%%%%%%%%%%%%%%%%%%%%%%%%%%%%%%%%%%%%%%%%%%%%%%%%%%%%%%%%%%%
% Paranal
%
%-----------------------------------------------------------------------
%---- CRIRES at the VLT-UT1 (ANTU) --------------------------------------
%-----------------------------------------------------------------------
%
%\INSconfig{}{CRIRES}{no-AO}{Provide list of reference wavelengths HERE}
% If you plan to use a NGS, please specify the NGS name in target list.
%\INSconfig{}{CRIRES}{NGS}{Provide list of reference wavelengths HERE}
%
%
%-----------------------------------------------------------------------
%---- FORS2 at the VLT-UT1 (ANTU) --------------------------------------
%-----------------------------------------------------------------------
%If you require the E2V detector please select this option as well as
%the required mode below.
%\INSconfig{}{FORS2}{Detector}{E2V}
%
%If you require the MIT detector please select this option as well as
%the required mode below.
%\INSconfig{}{FORS2}{Detector}{MIT}
%
%\INSconfig{}{FORS2}{collimator}{HR}
%\INSconfig{}{FORS2}{PRE-IMG}{ESO filters: provide list HERE}
%\INSconfig{}{FORS2}{IMG}{ESO filters: provide list HERE}
%\INSconfig{}{FORS2}{IMG}{User's own filters (to be described in text)}
%\INSconfig{}{FORS2}{IPOL}{ESO filters: provide list HERE}
%\INSconfig{}{FORS2}{IPOL}{User's own filters (to be described in text)}
%\INSconfig{}{FORS2}{LSS}{Provide list of grism+filter combinations HERE}
%\INSconfig{}{FORS2}{MOS}{Provide list of grism+filter combinations HERE}
%\INSconfig{}{FORS2}{PMOS}{Provide list of grism+filter combinations HERE}
%\INSconfig{}{FORS2}{MXU}{Provide list of grism+filter combinations HERE}
%\INSconfig{}{FORS2}{HITI}{ESO filters: provide list HERE}
%\INSconfig{}{FORS2}{HIT-OS}{Provide list of grisms HERE}
%\INSconfig{}{FORS2}{HIT-MS}{Provide list of grisms HERE}
%\INSconfig{}{FORS2}{RRM}{yes}
%
%-----------------------------------------------------------------------
%---- KMOS at the VLT-UT1 (ANTU) ---------------------------------------
%-----------------------------------------------------------------------
%
%\INSconfig{}{KMOS}{IFU}{provide list of settings (IZ, YJ, H, K, HK) here} 
%
%-----------------------------------------------------------------------
%---- UVES at the VLT-UT2 (KUEYEN) -------------------------------------
%-----------------------------------------------------------------------
%
%\INSconfig{}{UVES}{BLUE}{Standard setting: 346}
%\INSconfig{}{UVES}{BLUE}{Standard setting: 437}
%\INSconfig{}{UVES}{BLUE}{Non-std setting: provide central wavelength  HERE}
%
%\INSconfig{}{UVES}{RED}{Standard setting: 520}
%\INSconfig{}{UVES}{RED}{Standard setting: 580}
%\INSconfig{}{UVES}{RED}{Standard setting: 600}
%\INSconfig{}{UVES}{RED}{Iodine cell standard setting: 600}
%\INSconfig{}{UVES}{RED}{Standard setting: 860}
%\INSconfig{}{UVES}{RED}{Non-std setting: provide central wavelength HERE}
%
%\INSconfig{}{UVES}{DIC-1}{Standard setting: 346+580}
%\INSconfig{}{UVES}{DIC-1}{Standard setting: 390+564}
%\INSconfig{}{UVES}{DIC-1}{Standard setting: 346+564}
%\INSconfig{}{UVES}{DIC-1}{Standard setting: 390+580}
%\INSconfig{}{UVES}{DIC-1}{Non-std setting: provide central wavelength HERE}
%
%\INSconfig{}{UVES}{DIC-2}{Standard setting: 437+860}
%\INSconfig{}{UVES}{DIC-2}{Standard setting: 346+860}
%\INSconfig{}{UVES}{DIC-2}{Standard setting: 390+860}
%
%\INSconfig{}{UVES}{DIC-2}{Standard setting: 437+760}
%\INSconfig{}{UVES}{DIC-2}{Standard setting: 346+760}
%\INSconfig{}{UVES}{DIC-2}{Standard setting: 390+760}
%\INSconfig{}{UVES}{DIC-2}{Non-std setting: provide central wavelength HERE}
%
%\INSconfig{}{UVES}{Field Derotation}{yes}
%\INSconfig{}{UVES}{Image slicer-1}{yes}
%\INSconfig{}{UVES}{Image slicer-2}{yes}
%\INSconfig{}{UVES}{Image slicer-3}{yes}
%\INSconfig{}{UVES}{Iodine cell}{yes}
%\INSconfig{}{UVES}{Longslit Filters}{Provide list of filters HERE}
%
%\INSconfig{}{UVES}{RRM}{yes}
%
%
%-----------------------------------------------------------------------
%---- FLAMES at the VLT-UT2 (KUEYEN) -----------------------------------
%-----------------------------------------------------------------------
%\INSconfig{}{FLAMES}{UVES}{Specify the UVES setup below}
%\INSconfig{}{FLAMES}{GIRAFFE-MEDUSA}{Specify the GIRAFFE setup below}
%\INSconfig{}{FLAMES}{GIRAFFE-IFU}{Specify the GIRAFFE setup below}
%\INSconfig{}{FLAMES}{GIRAFFE-ARGUS}{Specify the GIRAFFE setup below}
%\INSconfig{}{FLAMES}{Combined: UVES + GIRAFFE-MEDUSA}{Specify the UVES and
%GIRAFFE setups below}
%\INSconfig{}{FLAMES}{Combined: UVES + GIRAFFE-IFU}{Specify the UVES and
%GIRAFFE setups below}
%\INSconfig{}{FLAMES}{Combined: UVES + GIRAFFE-ARGUS}{Specify the UVES and
%GIRAFFe setups below}
%
%
% If you have selected UVES, either standalone or in combined mode,
% please specify the UVES standard setup(s) to be used
%\INSconfig{}{FLAMES}{UVES}{standard setup Red 520}
%\INSconfig{}{FLAMES}{UVES}{standard setup Red 580}
%\INSconfig{}{FLAMES}{UVES}{standard setup Red 580 + simultaneous calibration}
%\INSconfig{}{FLAMES}{UVES}{standard setup Red 860}
%
%\INSconfig{}{FLAMES}{GIRAFFE}{fast readout mode 625kHz VM only}
%
% If you have selected GIRAFFE, either standalone or in combined mode
% please specify the GIRAFFE standard setups(s) to be used
%\INSconfig{}{FLAMES}{GIRAFFE}{standard setup HR01 379.0}
%\INSconfig{}{FLAMES}{GIRAFFE}{standard setup HR02 395.8}
%\INSconfig{}{FLAMES}{GIRAFFE}{standard setup HR03 412.4}
%\INSconfig{}{FLAMES}{GIRAFFE}{standard setup HR04 429.7}
%\INSconfig{}{FLAMES}{GIRAFFE}{standard setup HR05 447.1 A}
%\INSconfig{}{FLAMES}{GIRAFFE}{standard setup HR05 447.1 B}
%\INSconfig{}{FLAMES}{GIRAFFE}{standard setup HR06 465.6}
%\INSconfig{}{FLAMES}{GIRAFFE}{standard setup HR07 484.5 A}
%\INSconfig{}{FLAMES}{GIRAFFE}{standard setup HR07 484.5 B}
%\INSconfig{}{FLAMES}{GIRAFFE}{standard setup HR08 504.8}
%\INSconfig{}{FLAMES}{GIRAFFE}{standard setup HR09 525.8 A}
%\INSconfig{}{FLAMES}{GIRAFFE}{standard setup HR09 525.8 B}
%\INSconfig{}{FLAMES}{GIRAFFE}{standard setup HR10 548.8}
%\INSconfig{}{FLAMES}{GIRAFFE}{standard setup HR11 572.8}
%\INSconfig{}{FLAMES}{GIRAFFE}{standard setup HR12 599.3}
%\INSconfig{}{FLAMES}{GIRAFFE}{standard setup HR13 627.3}
%\INSconfig{}{FLAMES}{GIRAFFE}{standard setup HR14 651.5 A}
%\INSconfig{}{FLAMES}{GIRAFFE}{standard setup HR14 651.5 B}
%\INSconfig{}{FLAMES}{GIRAFFE}{standard setup HR15 665.0}
%\INSconfig{}{FLAMES}{GIRAFFE}{standard setup HR15 679.7}
%\INSconfig{}{FLAMES}{GIRAFFE}{standard setup HR16 710.5}
%\INSconfig{}{FLAMES}{GIRAFFE}{standard setup HR17 737.0 A}
%\INSconfig{}{FLAMES}{GIRAFFE}{standard setup HR17 737.0 B}
%\INSconfig{}{FLAMES}{GIRAFFE}{standard setup HR18 769.1}
%\INSconfig{}{FLAMES}{GIRAFFE}{standard setup HR19 805.3 A}
%\INSconfig{}{FLAMES}{GIRAFFE}{standard setup HR19 805.3 B}
%\INSconfig{}{FLAMES}{GIRAFFE}{standard setup HR20 836.6 A}
%\INSconfig{}{FLAMES}{GIRAFFE}{standard setup HR20 836.6 B}
%\INSconfig{}{FLAMES}{GIRAFFE}{standard setup HR21 875.7}
%\INSconfig{}{FLAMES}{GIRAFFE}{standard setup HR22 920.5 A}
%\INSconfig{}{FLAMES}{GIRAFFE}{standard setup HR22 920.5 B}
%\INSconfig{}{FLAMES}{GIRAFFE}{standard setup LR01 385.7}
%\INSconfig{}{FLAMES}{GIRAFFE}{standard setup LR02 427.2}
%\INSconfig{}{FLAMES}{GIRAFFE}{standard setup LR03 479.7}
%\INSconfig{}{FLAMES}{GIRAFFE}{standard setup LR04 543.1}
%\INSconfig{}{FLAMES}{GIRAFFE}{standard setup LR05 614.2}
%\INSconfig{}{FLAMES}{GIRAFFE}{standard setup LR06 682.2}
%\INSconfig{}{FLAMES}{GIRAFFE}{standard setup LR07 773.4}
%\INSconfig{}{FLAMES}{GIRAFFE}{standard setup LR08 881.7}
%
%\INSconfig{}{FLAMES}{GIRAFFE}{fast readout mode 625kHz VM only}
%
%-----------------------------------------------------------------------
%---- X-SHOOTER at the VLT-UT2 (KUEYEN) -----------------------------------
%-----------------------------------------------------------------------
%
%\INSconfig{}{XSHOOTER}{300-2500nm}{SLT}
%\INSconfig{}{XSHOOTER}{300-2500nm}{IFU}
%
%\INSconfig{}{XSHOOTER}{RRM}{yes}
%%%
% Slits (SLT only):
%
%UVB arm, available slits in arcsec: 0.5, 0.8, 1.0, 1.3, 1.6, 5.0
%VIS arm, available slits in arcsec: 0.4, 0.7, 0.9, 1.2, 1.5, 5.0 
%NIR arm, available slits in arcsec: 0.4, 0.6, 0.6JH, 0.9, 0.9JH, 1.2, 5.0
%  The 0.6JH and 0.9JH include a stray light K-band blocking filter
%  that allow sky limited studies in J and H bands.
%
%The slits for IFU  are fixed and do not need to be mentioned here.
%
% Replace SLIT_UVB, SLIT_VIS, SLIT_NIR with the choice of the slits:
%\INSconfig{}{XSHOOTER}{SLT}{SLIT_UVB,SLIT_VIS,SLIT_NIR}
%%%%%
% Detector readout mode:
%
% UVB and VIS arms: available readout modes and binning:
% 100k-1x1, 100k-1x2, 100k-2x2, 400k-1x1, 400k-1x2, 400k-2x2
% The NIR readout mode is fixed  to NDR.
%
%\INSconfig{}{XSHOOTER}{IFU}{readout UVB,readout VIS,readout NIR}
%\INSconfig{}{XSHOOTER}{SLT}{readout UVB,readout VIS,readout NIR}
%
%
%-----------------------------------------------------------------------
%---- ISAAC at the VLT-UT3 (MELIPAL) --------------------------------------
%-----------------------------------------------------------------------
%
%\INSconfig{}{ISAAC}{PRE-IMG}{Provide list of filters HERE}
%\INSconfig{}{ISAAC}{Hawaii Imaging}{Provide list of filters HERE}
%\INSconfig{}{ISAAC}{Aladdin Imaging}{Provide list of filters HERE}
%\INSconfig{}{ISAAC}{BURST}{Provide list of filters HERE}
%\INSconfig{}{ISAAC}{FASTJITT}{Provide list of filters HERE}
%\INSconfig{}{ISAAC}{SWS-LR}{Provide central wavelength(s) HERE}
%\INSconfig{}{ISAAC}{SWS-MR}{Provide central wavelength(s) HERE}
%\INSconfig{}{ISAAC}{LWS-LR}{Provide central wavelength(s) HERE}
%\INSconfig{}{ISAAC}{LWS-MR}{Provide central wavelength(s) HERE}
%\INSconfig{}{ISAAC}{SWP}{Provide list of filters HERE}
%\INSconfig{}{ISAAC}{RRM}{yes}
%
%
%
%-----------------------------------------------------------------------
%---- VIMOS at the VLT-UT3 (MELIPAL) -----------------------------------
%-----------------------------------------------------------------------
%
%\INSconfig{}{VIMOS}{PRE-IMG}{ESO filters: enter the list of filters}
%\INSconfig{}{VIMOS}{IMG}{ESO filters: enter the list of filters}
%\INSconfig{}{VIMOS}{IFU 0.67"/fibre}{LR-Red}
%\INSconfig{}{VIMOS}{IFU 0.67"/fibre}{LR-Blue}
%\INSconfig{}{VIMOS}{IFU 0.67"/fibre}{MR}
%\INSconfig{}{VIMOS}{IFU 0.67"/fibre}{HR-Red}
%\INSconfig{}{VIMOS}{IFU 0.67"/fibre}{HR-Orange}
%\INSconfig{}{VIMOS}{IFU 0.67"/fibre}{HR-Blue}
%
%\INSconfig{}{VIMOS}{IFU 0.33"/fibre}{LR-Red}
%\INSconfig{}{VIMOS}{IFU 0.33"/fibre}{LR-Blue}
%\INSconfig{}{VIMOS}{IFU 0.33"/fibre}{MR}
%\INSconfig{}{VIMOS}{IFU 0.33"/fibre}{HR-Red}
%\INSconfig{}{VIMOS}{IFU 0.33"/fibre}{HR-Orange}
%\INSconfig{}{VIMOS}{IFU 0.33"/fibre}{HR-Blue}
%
%\INSconfig{}{VIMOS}{MOS-grisms}{LR-Red}
%\INSconfig{}{VIMOS}{MOS-grisms}{LR-Blue}
%\INSconfig{}{VIMOS}{MOS-grisms}{MR}
%\INSconfig{}{VIMOS}{MOS-grisms}{HR-Red}
%\INSconfig{}{VIMOS}{MOS-grisms}{HR-Orange}
%\INSconfig{}{VIMOS}{MOS-grisms}{HR-Blue}
%
%\INSconfig{}{VIMOS}{MOS-slits-targets}{0.6" < slit width < 1.4", targets:stellar}
%\INSconfig{}{VIMOS}{MOS-slits-targets}{0.6" < slit width < 1.4", targets:extended}
%\INSconfig{}{VIMOS}{MOS-slits-targets}{slit width > 1.4", targets:stellar}
%\INSconfig{}{VIMOS}{MOS-slits-targets}{slit width > 1.4", targets:extended}
%\INSconfig{}{VIMOS}{MOS-masks}{Enter here number of mask sets (1 set = 4 quadrants)}
%
%
%-----------------------------------------------------------------------
%---- NAOS/CONICA at the VLT-UT4 (YEPUN) -------------------------------
%-----------------------------------------------------------------------
%
%\INSconfig{}{NACO}{PRE-IMG}{provide list of filters HERE}
%
% If you plan to use a NGS, please specify the NGS name, distance from target and magnitude  
%(Vmag preferred, otherwise Rmag) in the target list,
% and uncomment the following line
%\INSconfig{}{NACO}{NGS}{-}
%
% If you plan to use the LGS, please specify the TTS name,distance from the target and magnitude 
% (Vmag preferred, otherwise Rmag) in the target list,
% and uncomment the following line
%\INSconfig{}{NACO}{LGS}{-}
%
%
% If you plan to use the LGS without a TTS (seeing enhancer mode) then
% please leave the TTS name blank in the target list,
% and uncomment the following line
%\INSconfig{}{NACO}{LGS-noTTS}{-}
%
%\INSconfig{}{NACO}{Special Cal}{Select if you have special calibrations}
%\INSconfig{}{NACO}{Pupil Track}{Select if you need pupil tracking mode}
%\INSconfig{}{NACO}{Cube}{Select if you need cube mode}
%
%\INSconfig{}{NACO}{SAM VIS-WFS}{Provide list of masks and filters HERE}
%\INSconfig{}{NACO}{SAM IR-WFS}{Provide list of masks and filters HERE}
%\INSconfig{}{NACO}{SAMPol VIS-WFS}{Provide list of masks and filters HERE}
%\INSconfig{}{NACO}{SAMPol IR-WFS}{Provide list of masks and filters HERE}
%
%\INSconfig{}{NACO}{IMG 54 mas/px IR-WFS}{provide list of filters HERE}
%\INSconfig{}{NACO}{IMG 27 mas/px IR-WFS}{provide list of filters HERE}
%\INSconfig{}{NACO}{IMG 13 mas/px IR-WFS}{provide list of filters HERE}
%\INSconfig{}{NACO}{IMG 54 mas/px VIS-WFS}{provide list of filters HERE}
%\INSconfig{}{NACO}{IMG 27 mas/px VIS-WFS}{provide list of filters HERE}
%\INSconfig{}{NACO}{IMG 13 mas/px VIS-WFS}{provide list of filters HERE}
%
%\INSconfig{}{NACO}{SDI+ 17 mas/px IR-WFS}{comments}
%\INSconfig{}{NACO}{SDI+ 17 mas/px VIS-WFS}{comments}
%
%\INSconfig{}{NACO}{CORONA 54 mas/px IR-WFS}{provide list of masks and filters HERE}
%\INSconfig{}{NACO}{CORONA 27 mas/px IR-WFS}{provide list of masks and filters HERE}
%\INSconfig{}{NACO}{CORONA 13 mas/px IR-WFS}{provide list of masks and filters HERE}
%\INSconfig{}{NACO}{CORONA 54 mas/px VIS-WFS}{provide list of masks and filters HERE}
%\INSconfig{}{NACO}{CORONA 27 mas/px VIS-WFS}{provide list of masks and filters HERE}
%\INSconfig{}{NACO}{CORONA 13 mas/px VIS-WFS}{provide list of masks and filters HERE}
%\INSconfig{}{NACO}{CORONA 4QPM IR-WFS}{provide list of filters (H,K) HERE}
%\INSconfig{}{NACO}{CORONA 4QPM VIS-WFS}{provide list of filters (H,K) HERE}
%
%\INSconfig{}{NACO}{CORONA AGPM VIS-WFS}{provide list of filters (L',NB_3.74,NB_4.05) HERE}
%\INSconfig{}{NACO}{CORONA AGPM IR-WFS}{provide list of filters (L',NB_3.74,NB_4.05) HERE}
%
%\INSconfig{}{NACO}{POL 54 mas/px IR-WFS}{provide list of filters HERE}
%\INSconfig{}{NACO}{POL 27 mas/px IR-WFS}{provide list of filters HERE}
%\INSconfig{}{NACO}{POL 13 mas/px IR-WFS}{provide list of filters HERE}
%\INSconfig{}{NACO}{POL 54 mas/px VIS-WFS}{provide list of filters HERE}
%\INSconfig{}{NACO}{POL 27 mas/px VIS-WFS}{provide list of filters HERE}
%\INSconfig{}{NACO}{POL 13 mas/px VIS-WFS}{provide list of filters HERE}
%
%\INSconfig{}{NACO}{APP 54 mas/px IR-WFS}{select Lp and/or NB_4.05}
%\INSconfig{}{NACO}{APP 27 mas/px IR-WFS}{select Lp and/or NB_4.05}
%\INSconfig{}{NACO}{APP 54 mas/px VIS-WFS}{select Lp and/or NB_4.05}
%\INSconfig{}{NACO}{APP 27 mas/px VIS-WFS}{select Lp and/or NB_4.05}
%
%\INSconfig{}{NACO}{SPEC IR-WFS}{provide the list of spectroscopic modes HERE}
\INSconfig{A}{NACO}{SPEC VIS-WFS}{S54 3 SK}
% % 
%
%-----------------------------------------------------------------------
%---- SINFONI at the VLT-UT4 (YEPUN) -----------------------------------
%-----------------------------------------------------------------------
%

%\INSconfig{}{SINFONI}{PRE-IMG}{provide list of setting(s) (J,H,K,H+K)}
%
%\INSconfig{}{SINFONI}{IFS 250mas/pix no-AO}{provide list of setting(s) (J,H,K,H+K) HERE}
%\INSconfig{}{SINFONI}{IFS 100mas/pix no-AO}{provide list of setting(s) (J,H,K,H+K) HERE}
%
% If you plan to use a NGS, please specify the NGS name and magnitude (Rmag preferred,
% otherwise Vmag) in target list.
%\INSconfig{}{SINFONI}{IFS 250mas/pix NGS}{provide list of setting(s) (J,H,K,H+K) HERE}
%\INSconfig{}{SINFONI}{IFS 100mas/pix NGS}{provide list of setting(s) (J,H,K,H+K) HERE}
%\INSconfig{}{SINFONI}{IFS 25mas/pix NGS}{provide list of setting(s) (J,H,K,H+K) HERE}
%
% If you plan to use the LGS, please specify the TTS name and magnitude (Rmag preferred,
% otherwise Vmag) in target list.
%\INSconfig{}{SINFONI}{IFS 250mas/pix LGS}{provide list of setting(s) (J,H,K,H+K) HERE}
%\INSconfig{}{SINFONI}{IFS 100mas/pix LGS}{provide list of setting(s) (J,H,K,H+K) HERE}
%\INSconfig{}{SINFONI}{IFS 25mas/pix LGS}{provide list of setting(s) (J,H,K,H+K) HERE}
%
% If you plan to use the LGS without a TTS (seeing enhancer mode) then
% please leave the TTS name blank in the target list.
%\INSconfig{}{SINFONI}{IFS 250mas/pix LGS-noTTS}{provide list of setting(s) (J,H,K,H+K) HERE}
%\INSconfig{}{SINFONI}{IFS 100mas/pix LGS-noTTS}{provide list of setting(s) (J,H,K,H+K) HERE}
%\INSconfig{}{SINFONI}{IFS 25mas/pix LGS-noTTS}{provide list of setting(s) (J,H,K,H+K) HERE}
%
%\INSconfig{}{SINFONI}{RRM}{yes}
%
%
%%-----------------------------------------------------------------------
%---- HAWKI at the VLT-UT4 (YEPUN) -----------------------------------
%-----------------------------------------------------------------------
%\INSconfig{}{HAWKI}{PRE-IMG}{provide list of filters (Y,J,H,Ks,CH4,BrG,H2,NB0984,NB1060,NB2090) HERE}
%\INSconfig{}{HAWKI}{IMG}{provide list of filters (Y,J,H,Ks,CH4,BrG,H2,NB0984,NB1060,NB2090) HERE}
%\INSconfig{}{HAWKI}{BURST}{Provide list of filters  (Y,J,H,Ks,CH4,BrG,H2,NB0984,NB1060,NB2090) HERE}
%\INSconfig{}{HAWKI}{FASTJITT}{Provide list of filters  (Y,J,H,Ks,CH4,BrG,H2,NB0984,NB1060,NB2090) HERE}
%\INSconfig{}{HAWKI}{RRM}{yes}
%
%-----------------------------------------------------------------------
%---- Interferometric Instruments --------------------------------------
%-----------------------------------------------------------------------
%
%-----------------------------------------------------------------------
%---- MIDI -------------------------------------------------------------
%-----------------------------------------------------------------------
%
%
%\INSconfig{}{MIDI}{PRISM}{CORR-FLUX}
%
% For MIDI + PRIMA - FSU
%\INSconfig{}{MIDI}{PRISM}{CORR-FLUX-F} 
%
%\INSconfig{}{MIDI}{PRISM}{HIGH-SENS}
%\INSconfig{}{MIDI}{GRISM}{HIGH-SENS}
%
%\INSconfig{}{MIDI}{PRISM}{SCI-PHOT}
%\INSconfig{}{MIDI}{GRISM}{SCI-PHOT}
%
%
%
%-----------------------------------------------------------------------
%---- AMBER ------------------------------------------------------------
%-----------------------------------------------------------------------
%
%\INSconfig{}{AMBER}{LR-HK-F}{2.2}
%\INSconfig{}{AMBER}{LR-HK}{2.2}
%\INSconfig{}{AMBER}{MR-K-F}{2.1}
%\INSconfig{}{AMBER}{MR-K}{2.1}
%\INSconfig{}{AMBER}{MR-H-F}{1.65} % << updated in P87
%\INSconfig{}{AMBER}{MR-H}{1.65}   % << updated in P87
%\INSconfig{}{AMBER}{MR-K-F}{2.3}
%\INSconfig{}{AMBER}{MR-K}{2.3}
%\INSconfig{}{AMBER}{HR-K}{Central wavelength selected from the list:
% 1.97929,2.01786,2.05643,2.09500,2.13357,2.17214,2.21071,2.24929,2.28786,2.32643,
% 2.36500,2.40357,2.44214,2.48071}
%\INSconfig{}{AMBER}{HR-K-F}{Central wavelength selected from the list:
% 1.97929,2.01786,2.05643,2.09500,2.13357,2.17214,2.21071,2.24929,2.28786,2.32643,
% 2.36500,2.40357,2.44214,2.48071}
% 
%where *-F means with FINITO
%
%-----------------------------------------------------------------------
%---- VIRCAM at VISTA --------------------------------------------------
%-----------------------------------------------------------------------
%
%\INSconfig{}{VIRCAM}{IMG}{provide list of filters here}
%
%-----------------------------------------------------------------------
%---- OMEGACAM at VST --------------------------------------------------
% This instrument is only available for GTO and Chilean programmes.
%-----------------------------------------------------------------------
%
%\INSconfig{}{OMEGACAM}{IMG}{provide list of filters here}
%
%%%%%%%%%%%%%%%%%%%%%%%%%%%%%%%%%%%%%%%%%%%%%%%%%%%%%%%%%%%%%%%%%%%%%%%%
% La Silla
%-----------------------------------------------------------------------
%---- EFOSC2 (or SOFOSC) at the NTT ------------------------------------
%-----------------------------------------------------------------------
%
%\INSconfig{}{EFOSC2}{PRE-IMG}{EFOSC2 filters: provide list here}
%\INSconfig{}{EFOSC2}{Imaging-filters}{EFOSC2 filters:  provide list here}
%\INSconfig{}{EFOSC2}{Imaging-filters}{ESO non EFOSC filters: provide ESOfilt No}
%\INSconfig{}{EFOSC2}{Imaging-filters}{User's own filters (to be described in text)}
%\INSconfig{}{EFOSC2}{Spectro-long-slit}{Grism\#1:320-1090}
%\INSconfig{}{EFOSC2}{Spectro-long-slit}{Grism\#2:510-1100}
%\INSconfig{}{EFOSC2}{Spectro-long-slit}{Grism\#3:305-610}
%\INSconfig{}{EFOSC2}{Spectro-long-slit}{Grism\#4:409-752}
%\INSconfig{}{EFOSC2}{Spectro-long-slit}{Grism\#5:520-935}
%\INSconfig{}{EFOSC2}{Spectro-long-slit}{Grism\#6:386-807}
%\INSconfig{}{EFOSC2}{Spectro-long-slit}{Grism\#7:327-524}
%\INSconfig{}{EFOSC2}{Spectro-long-slit}{Grism\#8:432-636}
%\INSconfig{}{EFOSC2}{Spectro-long-slit}{Grism\#11:338-752}
%\INSconfig{}{EFOSC2}{Spectro-long-slit}{Grism\#13:369-932}
%\INSconfig{}{EFOSC2}{Spectro-long-slit}{Grism\#14:310-509}
%\INSconfig{}{EFOSC2}{Spectro-long-slit}{Grism\#16:602-1032}
%\INSconfig{}{EFOSC2}{Spectro-long-slit}{Grism\#17:689-876}
%\INSconfig{}{EFOSC2}{Spectro-long-slit}{Grism\#18:470-677}
%\INSconfig{}{EFOSC2}{Spectro-long-slit}{Grism\#19:440-510}
%\INSconfig{}{EFOSC2}{Spectro-long-slit}{Grism\#20:605:715}
%\INSconfig{}{EFOSC2}{Spectro-long-slit}{Aperture: 0.5'', ... ,10.0''}
%
%\INSconfig{}{EFOSC2}{Spectro-long-slit}{Aperture: Shiftable}
%\INSconfig{}{EFOSC2}{Spectro-MOS}{PunchHead=0.95''}
%\INSconfig{}{EFOSC2}{Spectro-MOS}{PunchHead=1.12''}
%\INSconfig{}{EFOSC2}{Spectro-MOS}{PunchHead=1.45''}
%\INSconfig{}{EFOSC2}{Polarimetry}{$\lambda / 2$ retarder plate}
%\INSconfig{}{EFOSC2}{Polarimetry}{$\lambda / 4$ retarder plate}
%\INSconfig{}{EFOSC2}{Coronograph}{yes}
%
%
%-----------------------------------------------------------------------
%---- SOFI (or SOFOSC) at the NTT --------------------------------------------------
%-----------------------------------------------------------------------
%
%\INSconfig{}{SOFI}{PRE-IMG-LargeField}{Provide list of filters HERE}
%\INSconfig{}{SOFI}{Imaging-LargeField}{Provide list of filters HERE}
%\INSconfig{}{SOFI}{Burst}{Provide list of filters HERE}
%\INSconfig{}{SOFI}{FastPhot}{Provide list of filters HERE}
%\INSconfig{}{SOFI}{Polarimetry}{Provide list of filters HERE}
%\INSconfig{}{SOFI}{Spectroscopy-long-slit}{Blue Grism, Provide list of slits HERE}
%\INSconfig{}{SOFI}{Spectroscopy-long-slit}{Red Grism, Provide list of slits HERE}
%\INSconfig{}{SOFI}{Spectroscopy-high-res}{H, Provide list of slits HERE}
%\INSconfig{}{SOFI}{Spectroscopy-high-res}{K, Provide list of slits HERE}
%
%
%-----------------------------------------------------------------------
%---- HARPS at the 3.6 -------------------------------------------------
%-----------------------------------------------------------------------
%
%\INSconfig{}{HARPS}{spectro-Thosimult}{HARPS}
%\INSconfig{}{HARPS}{WAVE}{HARPS}
%\INSconfig{}{HARPS}{spectro-ObjA(B)}{HARPS}
%\INSconfig{}{HARPS}{spectro-ObjA(B)}{EGGS}
%\INSconfig{}{HARPS}{spectro-polarimetry}{linear}
%\INSconfig{}{HARPS}{spectro-polarimetry}{circular}
%
%
%-----------------------------------------------------------------------
%---- WFI at the ESO 2.2-m ---------------------------------------------
%-----------------------------------------------------------------------
%
%\INSconfig{}{WFI}{PRE-IMG}{provide WFI filter names HERE}
%\INSconfig{}{WFI}{imaging}{provide WFI filter names HERE}
%
%
%-----------------------------------------------------------------------
%---- FEROS at the ESO 2.2-m -------------------------------------------
%-----------------------------------------------------------------------
%
%\INSconfig{}{FEROS}{350-920nm}{OBJ-SKY, ADC}
%\INSconfig{}{FEROS}{350-920nm}{OBJ-SKY, no ADC}
%\INSconfig{}{FEROS}{350-920nm}{OBJ-CAL}
%
%
%%%%%%%%%%%%%%%%%%%%%%%%%%%%%%%%%%%%%%%%%%%%%%%%%%%%%%%%%%%%%%%%%%%%%%%%
% Chajnantor
%-----------------------------------------------------------------------
%---- SHFI at APEX ----------------------------------------------
%-----------------------------------------------------------------------
%
%\INSconfig{}{SHFI}{APEX-1}{Please enter Central Frequency 211 to 275 GHz}
%\INSconfig{}{SHFI}{APEX-2}{Please enter Central Frequency 275 to 370 GHz}
%\INSconfig{}{SHFI}{APEX-3}{Please enter Central Frequency 385 to 500 GHz} 
%\INSconfig{}{SHFI}{APEX-T2}{Please enter Central Frequency 1.25 to 1.39 THz}
%
%-----------------------------------------------------------------------
%---- LABOCA at APEX ----------------------------------------------
%-----------------------------------------------------------------------
%
%\INSconfig{}{LABOCA}{IMG}{-}
%\INSconfig{}{LABOCA}{PHOT}{-}
%
%-----------------------------------------------------------------------
%---- SABOCA at APEX ----------------------------------------------
%-----------------------------------------------------------------------
%
%\INSconfig{}{SABOCA}{IMG}{-}
%\INSconfig{}{SABOCA}{PHOT}{-}
%
%-----------------------------------------------------------------------
%---- FLASH at APEX ----------------------------------------------
%-----------------------------------------------------------------------
%
%\INSconfig{}{FLASH}{-}{Please enter Central Frequency 272 to 377 GHz and 385 to 495 GHz}
%
%-----------------------------------------------------------------------
%---- CHAMP+ at APEX ----------------------------------------------
%-----------------------------------------------------------------------
%
%\INSconfig{}{CHAMPP}{-}{Please enter Central Frequency 620 to 729 GHz and 780 to 900 GHz}
%-----------------------------------------------------------------------



%%%%%%%%%%%%%%%%%%%%%%%%%%%%%%%%%%%%%%%%%%%%%%%%%%%%%%%%%%%%%%%%%%%%%%%%
%%%%% Interferometry PAGE %%%%%%%%%%%%%%%%%%%%%%%%%%%%%%%%%%%%%%%%%%%%%%
%%%%%%%%%%%%%%%%%%%%%%%%%%%%%%%%%%%%%%%%%%%%%%%%%%%%%%%%%%%%%%%%%%%%%%%%
%
% The \VLTITarget macro is only needed when requesting
% Interferometry, in which case it is MANDATORY to uncomment it and
% fill in the information. It takes the following parameters:
%
% 1st argument: run ID
% Valid values: run IDs specified in BOX 3
%
% 2nd argument: target name
% This parameter is NOT checked at the pdfLaTeX compilation.
%
% 3rd argument: visual magnitude
% Values with up to decimal places are allowed here.
% This parameter is NOT checked at the pdfLaTeX compilation.
%
% 4th argument: magnitude at wavelength of observation
% Values with up to decimal places are allowed here.
% This parameter is NOT checked at the pdfLaTeX compilation.
%
% 5th argument: wavelength of observation (in microns)
% Values with up to decimal places are allowed here.
% This parameter is NOT checked at the pdfLaTeX compilation.
%
% 6th argument: size at wavelength of observation (in mas)
% This parameter is NOT checked at the pdfLaTeX compilation.
%
% 7th argument: baseline
% UT observations are scheduled in terms of 3-telescope 
% baselines for AMBER and 2-telescope baselines for MIDI.
% For UT observations please specify one of the four available 
% AMBER baselines or one of the six available MIDI baselines.
%
% AT observations are scheduled in terms of 4-telescope 
% configurations (quadruplets). For these observations, the 
% time can be split among the different 3-telescope baselines 
% (for AMBER) or 2-telescope baselines (for MIDI); the exact 
% baselines will be specified at Phase 2.
% For AT observations, please specify only one of the 3 
% available AT quadruplets at this stage.
%
%
% 8th parameter: visibility for the specified configuration
% (at preferred hour angle or hour angle 0)
% This parameter is NOT checked at the pdfLaTeX compilation.
%
% For AMBER observations, please specify the three visibility
% values corresponding to the three baselines of the chosen 
% VLTI configurations, separated by "/"; up to two of these 
% values may be replaced by '*'.
% This parameter is NOT checked at the pdfLaTeX compilation.
%
% For AT observations, please use one typical baseline of the 
% quadruplet that you have specified in order to compute
% typical visibility values.
%
%
% 9th parameter: correlated magnitude
% (for the visibility values specified in the 8th parameter)
% This parameter is NOT checked at the pdfLaTeX compilation.
%
% 10th parameter: time on target in hours
% Values with up to decimal places are allowed here.
% This parameter is NOT checked at the pdfLaTeX compilation.
%
% Note: For MIDI observations in any mode, please indicate 10.6 as
% wavelength of observation.
%
% The available baselines for Period 92 are shown below.
% For AT observations, the time can be split at Phase 2 among the
% different 2-telescope (for MIDI) or 3-telescope (AMBER) baselines
% of the chosen quadruplet. All possible 2-telescope (MIDI) or
% 3-telescope(AMBER) baselines available in these quadruplets
% are offered in both service and visitor mode.
%
% 
% AMBER
% A1-B2-C1-D0
% A1-G1-K0-J3
% D0-H0-G1-I1
% UT1-UT2-UT3
% UT1-UT2-UT4
% UT1-UT3-UT4
% UT2-UT3-UT4
% 
% MIDI
% A1-B2-C1-D0
% A1-G1-K0-J3
% D0-H0-G1-I1
% UT1-UT2-57m
% UT1-UT3-102m
% UT1-UT4-130m
% UT2-UT3-47m
% UT2-UT4-89m
% UT3-UT4-62m
% 
% SpecialVLTI
% A1-B2-C1-D0
% A1-G1-K0-J3
% D0-H0-G1-I1
% UT1-UT2
% UT1-UT2-UT3
% UT1-UT2-UT3-UT4
% UT1-UT2-UT4
% UT1-UT3
% UT1-UT3-UT4
% UT1-UT4
% UT2-UT3
% UT2-UT3-UT4
% UT2-UT4
% UT3-UT4
% 

%\VLTITarget{E}{Alpha Ori}{-1.4}{-1.4}{10.6}{6}{UT1-UT2-UT3}{0.45/0.60/0.10}{0.3/-0.2/4.0}{2}
%\VLTITarget{F}{Alpha Ori}{-1.4}{-1.4}{10.6}{6}{D0-H0-G1-I1}{0.80}{-0.9}{1}

% You can specify here a note applying to all or some of your VLTI
% targets.  You should take advantage of this note to indicate
% suitable alternative baselines for your observations.
% This macro is NOT checked at the pdfLaTeX compilation.

%\VLTITargetNotes{Note about the VLTI targets, e.g., Run E can also be carried out using UT1-UT3-UT4.}


%%%%%%%%%%%%%%%%%%%%%%%%%%%%%%%%%%%%%%%%%%%%%%%%%%%%%%%%%%%%%%%%%%%%%%%%
%%%%% ToO PAGE %%%%%%%%%%%%%%%%%%%%%%%%%%%%%%%%%%%%%%%%%%%%%%%%%%%%%%%%%
%%%%%%%%%%%%%%%%%%%%%%%%%%%%%%%%%%%%%%%%%%%%%%%%%%%%%%%%%%%%%%%%%%%%%%%%
%
% The \ToOrun macro is needed only when requesting Target of
% Opportunity (ToO) observations, in which case it is MANDATORY to
% uncomment it and fill in the information. It takes the following
% parameters: 
%
% 1st argument: run ID
% Valid values: run IDs specified in BOX 3
%
% 2nd argument: nature of observation
% Valid values: RRM, ToO-hard, ToO-soft
%
% 3rd argument: number of targets per run
% This parameter is NOT checked at the pdfLaTeX compilation.
%
% 4th argument: number of triggers per targets
% (for RRM and ToO observations only)
% This parameter is NOT checked at the pdfLaTeX compilation.

%\TOORun{A}{RRM}{2}{3}
%\TOORun{B}{ToO-hard}{3}{1}

% You have the opportunity to add notes to the ToO runs by using
% the \TOONotes macro.
% This macro is NOT checked at the pdfLaTeX compilation.

%\TOONotes{Use this macro to add a note to the ToO page.}


%%%%%%%%%%%%%%%%%%%%%%%%%%%%%%%%%%%%%%%%%%%%%%%%%%%%%%%%%%%%%%%%%%%%%%%%
%%%%% VISITOR SPECIAL INSTRUMENT PAGE %%%%%%%%%%%%%%%%%%%%%%%%%%%%%%%%%%
%%%%%%%%%%%%%%%%%%%%%%%%%%%%%%%%%%%%%%%%%%%%%%%%%%%%%%%%%%%%%%%%%%%%%%%%
%
% The following commands are only needed when bringing a Visitor
% Special Instrument, in which case it is MANDATORY to uncomment them
% and provide all the required information.
%
%\Desc{}   %Description of the instrument and its operation
%\Comm{}   %On which telescope(s) has instrument been commissioned/used
%\WV{}     %Total weight and value of equipment to be shipped
%\Wfocus{} %Weight at the focus (including ancillary equipment)
%\Interf{} %Compatibility of attachment interface with required focus
%\Focal{}  %Back focal distance value
%\Acqu{}   %Acquisition, focusing, and guiding procedure
%\Softw{}  %Compatibility with ESO software standards (data handling)
%\Suppl{}  %Estimate of services expected from ESO (in person days)

%%%%%%%%%%%%%%%%%%%%%%%%%%%%%%%%%%%%%%%%%%%%%%%%%%%%%%%%%%%%%%%%%%%%%%%%
%%%%% THE END %%%%%%%%%%%%%%%%%%%%%%%%%%%%%%%%%%%%%%%%%%%%%%%%%%%%%%%%%%
%%%%%%%%%%%%%%%%%%%%%%%%%%%%%%%%%%%%%%%%%%%%%%%%%%%%%%%%%%%%%%%%%%%%%%%%
\MakeProposal
\end{document}


